\documentclass[12pt]{article}

\usepackage{sbc-template}

\usepackage{graphicx}

\usepackage{url}

\usepackage{algorithms}

\usepackage{algorithmic}

\usepackage[brazil]{babel}   
%\usepackage[latin1]{inputenc}  
\usepackage[utf8]{inputenc}  
% UTF-8 encoding is recommended by ShareLaTex

     
\sloppy

\title{Sistema de Gerenciameto de Imunizados}

\author{Lucas A. da Encarnação Oliveira\inst{1} }


\address{Departamento de Ciências Exatas -- Universidade Estadual de Feira de Santana
  (UEFS)\\
  Feira de Santana -- BA -- Brazil
  \email{lucas2213690@gmail.com}
}

\begin{document} 

\maketitle
     
\begin{resumo} 
  Este artigo irá abordar um Sistema de Gerenciamento de Imunizados, para fins
  educacionais. Implementado em linguagem C com as funcionalidades de cadastro,
  consulta, atualização, remoção, e exibição dos dados de pacientes de determinada
  idade em ordem alfabética.
\end{resumo}


\section{Introdução}

A Tecnologia da Informação é muito utilizada em Sistemas de Gerenciamento,
pela facilidade em organizar, acessar e mudar dados e informações. Deste modo
este presente relatório traz um Sistema de Gerenciamento de Imunizados de H1N1, implementado
de forma dinâmica, através da estrutura de dados de lista encadeada.

\section{Metodologia} \label{sec:firstpage}
   

\subsection{Sessões Tutoriais}

Foi claro a necessidade de uso da lista encadeada como estrutura de dados
para este projeto, já que deve ser gerenciado dinâmicamente, para isso também
surge a necessidade de utilizar uma estrutura, já adicionando os campos de cadastro.
Na inserção surgiu a ideia de utilizar um elemento estático como início da lista
e o método de inserir no final da lista, evitando assim tratar exceções na primeira
inicialização e na remoção. Outra questão levantada é o uso de menus e sub-menus para
informar o usuário sobre as diversas opçãos do sistema. Na remoção também é interessante
verificar se o elemento é o primeiro ou último, caso o primeiro elemento não seja 
estático. Indagou-se qual método de ordenação implementar para exibição em
ordem alfabética.

\subsection{Implementação}

Utilizou-se uma estrutura com os campos de cadastro e lista encadeada simples, pois não havia necessidade de lista duplamente
encadeada. Implementou-se a ideia de um elemento estático e inserção no final, evitando exceções na manipulação da lista.
Utilizou-se menus e sub-menus intuitivos na interface. Como foi implementado um elemento inicial estático, não houve necessidade
de tratar exceções na manipulação da lista.      

\subsection{Requisitos}

Os requisitos do sistema de gerenciamento são:\\

\begin{itemize}
\item Cadastro de pessoas vacinadas com informações pessoais
\item Consulta de dados através do número de cadastro, nome e rg
\item Atualização de dados
\item Remoção de cadastro e exibição de dados em ordem crescente de pacientes com determinada idade
\end{itemize}

\subsection{Ambiente Utilizado}

No desenvolvimento do projeto foi utilizado o Sistema Operacional Windows 10.0.14332 64 Bits e
o GCC 

\section{Fundamentação Teórica}

\subsection{STDLIB.H}

O padrão ANSI C normaliza que os prótotipos para funções de alocação dinâmica
estão contidas na biblioteca STDLIB.H

\subsubsection{Função sizeof}
 
  \cite{kernighan1988c}, define a função sizeof como:
\begin{quotation}

``A linguagem C possui um operador unário em tempo de execução que pode ser usado para calcular
o tamanho de qualquer objeto. Essa expressão produz um inteiro igual ao tamanho do 
objeto ou tipo em bytes. Um objeto pode ser variavel ou vetor ou estrutura. Um tipo de nome
pode ser um tipo básico como int ou double, ou um tipo derivado de uma estrutura ou um ponteiro''.
\end{quotation}



\subsubsection{Função malloc}

%void *malloc(size_t size)\newline \newline
Recebe como argumento o número de bytes de determinado tipo de dados, alocando memória suficiente no ``heap'' em tempo
de execução, tem como retorno o endereço do primeiro byte \cite{schildt:96}.

\subsubsection{Função free}

void free(void *ptr)\\
Recebe como argumento um ponteiro e devolve ao heap o endereço apontado por esse argumento, liberando a memória.
Somente deve ser utilizada para liberar um espaço breviamente alocado dinamicamente\cite{schildt:96}. 

\subsection{STDBOOL.H}

Introduzida através norma ISO 9899:1999 (C99) no ANSI C, possibilita o uso do tipo de dado Bool e define macros como: ``true'' e ``false'' \cite{stroustrup2002sibling}. 

\subsection{STRING.H}

\subsubsection{Função strcpm}

int strcmp ( const char * str1, const char * str2 )\\

Tem como parâmetro o ponteiro de duas string, desse modo realiza a comparação. Retorna um valor negativo caso o primeiro argumento
tiver um valor menor, valor positivo caso o primeiro argumento será maior e zero se as duas strings forem iguais \cite{kernighan1988c}.



\section{Resultado e Discussões}

\subsection{Menu Principal}

Apresenta o menu principal com as diversas opções baseadas nos requisitos do projeto,
assim retorna a opção escolhida, para executar determinada função 


\subsection{Menu Consulta}

Exibe um sub-menu, com as diversas opção de busca para que seja executado um determinado
método


\subsection{Leitura}

Recebe um ponteiro do tipo lista, e pede a leitura de dados como nome, idade, 
endereço, posto de saúde, salvando estes dados nos campos correspondentes
da estrutura

\subsection{Função Inserção Final}


\begin{algorithm}                      % enter the algorithm environment
\caption{Inserção no final da lista}          % give the algorithm a caption
\label{alg1}                           % and a label for \ref{} commands later in the document
\begin{algorithmic}                    % enter the algorithmic environment
    \STATE $old \Leftarrow LISTA\hat.proximo$
    \STATE $novo \Leftarrow new(elo)$
    \STATE $leitura(novo)$
    \STATE $LISTA\hat.proximo \Leftarrow novo$
    \STATE $novo\hat.proximo \Leftarrow old$
\end{algorithmic}
\end{algorithm}

Esta função coloca um item dinamicamente no final da lista, como descrito no algoritmo \ref{alg1}.



\subsection{Função Atualizar}

Esta função recebe o número de cadastro que o usuário deseja atualizar,
exibe os dados antigos e refaz a leitura, atualizando os dados. Caso
o cadastro não seja encontrado, emite uma mensagem informativa


\begin{algorithm}                      % enter the algorithm environment
\caption{Atualização dados}          % give the algorithm a caption
\label{alg2}                           % and a label for \ref{} commands later in the document
\begin{algorithmic}                    % enter the algorithmic environment
    \STATE $achou \Leftarrow \TRUE$
    \STATE $aux \Leftarrow LISTA\hat.proximo$
    \WHILE {$((aux \neq nill) \AND (achou \neq \TRUE))$}
      \IF{$(aux\hat.numCadastro = cadastro)$}
        \STATE $exibir(aux)$
        \STATE $leitura(aux)$
        \STATE $achou \Leftarrow \TRUE$
      \ENDIF  
    \STATE $aux \Leftarrow aux\hat.proximo$
    \ENDWHILE
    \IF{$achou \neq \TRUE$}
      \PRINT \texttt{"Nao Encontrado"}
    \ENDIF
\end{algorithmic}
\end{algorithm}  


\subsection{Consulta Cadastro}

Esta função recebe como parâmetro de entrada o número de cadastro procurado, assim
através de um loop while,como descrito no algoritmo \ref{alg2} percorre a lista e ao encontrar o cadastro desejado, exibe
todos seus dados e pede novos dados. Caso não encontre, emite uma mensagem de alerta. 

\subsection{Consulta Nome}

Implementa o mesmo mecanismo da função ```Consulta Cadastro'', sendo que o único diferencial
é que a busca é guiada pelo nome de cadastro.

\subsection{Consulta RG}

Também possui muitas semelhanças com as duas funções acima, mas a busca é guiada pelo número
do Registro Geral cadastrado.

\subsection{Remover}

A função remover procura através do loop while, o cadastro que o usuário deseja remover, sempre
guardando os valores do próximo e anterior. Ao encontrar, o espaço de memória desse cadastro é
liberado e seu anterior e próximo são ligados através do uso de ponteiros, para que a lista continue
mesmo após sua remoção, como descrito no algoritmo \ref{alg3}

\begin{algorithm}                      % enter the algorithm environment
\caption{Remoção}          % give the algorithm a caption
\label{alg3}                           % and a label for \ref{} commands later in the document
\begin{algorithmic}                    % enter the algorithmic environment
    \STATE $anterior \Leftarrow LISTA$
    \STATE $aux \Leftarrow LISTA\hat.proximo$
    \WHILE {$((aux \neq nill) \AND (aux\hat.numCadastro \neq cadastro))$}
        \STATE $anterior \Leftarrow aux$
        \STATE $aux \Leftarrow aux\hat.proximo$
    \ENDWHILE
    \IF {$(aux \neq nill)$}
      \STATE $anterior\hat.proximo \Leftarrow aux\hat.proximo$
      \STATE $dispose(aux)$
    \ELSE
      \PRINT \texttt{"Nao Encontrado"}
    \ENDIF
\end{algorithmic}
\end{algorithm}  

\subsection{Ordem alfabética}
Recebe um matriz de caracteres, que representa os nomes de imunizados e através de um laço for e while reorganiza por ordem
alfabética, como descrito no algoritmo \ref{alg4}.


\begin{algorithm}                      % enter the algorithm environment
\caption{Ordem Alfabética}          % give the algorithm a caption
\label{alg4}                           % and a label for \ref{} commands later in the document
\begin{algorithmic}                    % enter the algorithmic environment
    \FOR {$j=1$ \TO $total$}
      \STATE $StrPCopy(aux, nomes[j])$
      \STATE $i \Leftarrow j-1$
      \WHILE {$((i \geq 0) \AND AnsiStrIComp(nomes[i],aux) > 0)$}
        \STATE $StrPCopy(nomes[i+1], nomes[i])$
        \STATE $i \Leftarrow i -1$
      \ENDWHILE
      \STATE $StrPCopy(nomes[i+1], aux)$
    \ENDFOR
\end{algorithmic}
\end{algorithm}

\subsection{Exibir}

Recebe um ponteiro correspondente a um nó da lista e exibe todos os seus campos,
que são dados de um paciente, como descrito no algoritmo.


\subsection{Relatório 2}

Esta função conta a quantidade de pacientes com determinada idade, cria um vetor dinamicamente
para armazenar seus nomes, chama a função ordem alfabetica e exibe os dados dos pacientes por ordem
alfabética como descrito no algoritmo \ref{alg5}

\begin{algorithm}                      % enter the algorithm environment
\caption{Exibir em ordem alfabética}          % give the algorithm a caption
\label{alg5}                           % and a label for \ref{} commands later in the document
\begin{algorithmic}
  \WHILE {$(i < total)$}                    % enter the algorithmic environment
      \WHILE {$((aux \neq nill)$}
          \IF {\NOT $AnsiStrIComp(aux\hat.nome,nomes[i])$}
            \STATE $exibir(aux))$
            \STATE $i \Leftarrow i+1$
          \ENDIF
      \ENDWHILE
      \STATE $aux \Leftarrow aux\hat.proximo$
  \ENDWHILE
\end{algorithmic}
\end{algorithm}  



\section{Conclusão}

Este trabalhou possibilitou a aprendizagem da manipulação da estrutura de lista encadeada,
implementada em um ambiente dinamico, onde é necessário o gerenciamento de recursos computacio
nais.

\bibliographystyle{sbc}
\bibliography{sbc-template}

\end{document}
