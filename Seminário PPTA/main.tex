% Copyright 2004 by Till Tantau <tantau@users.sourceforge.net>.
%
% In principle, this file can be redistributed and/or modified under
% the terms of the GNU Public License, version 2.
%
% However, this file is supposed to be a template to be modified
% for your own needs. For this reason, if you use this file as a
% template and not specifically distribute it as part of a another
% package/program, I grant the extra permission to freely copy and
% modify this file as you see fit and even to delete this copyright
% notice. 

\documentclass{beamer}

\usepackage[brazil]{babel}   
%\usepackage[latin1]{inputenc}  
\usepackage[utf8]{inputenc}  
% UTF-8 encoding is recommended by ShareLaTex

% There are many different themes available for Beamer. A comprehensive
% list with examples is given here:
% http://deic.uab.es/~iblanes/beamer_gallery/index_by_theme.html
% You can uncomment the themes below if you would like to use a different
% one:
%\usetheme{AnnArbor}
%\usetheme{Antibes}
%\usetheme{Bergen}
%\usetheme{Berkeley}
%\usetheme{Berlin}
%\usetheme{Boadilla}
%\usetheme{boxes}
%\usetheme{CambridgeUS}
%\usetheme{Copenhagen}
%\usetheme{Darmstadt}
%\usetheme{default}
%\usetheme{Frankfurt}
%\usetheme{Goettingen}
%\usetheme{Hannover}
%\usetheme{Ilmenau}
%\usetheme{JuanLesPins}
%\usetheme{Luebeck}
\usetheme{Madrid}
%\usetheme{Malmoe}
%\usetheme{Marburg}
%\usetheme{Montpellier}
%\usetheme{PaloAlto}
%\usetheme{Pittsburgh}
%\usetheme{Rochester}
%\usetheme{Singapore}
%\usetheme{Szeged}
%\usetheme{Warsaw}

\title{Utilização do método PBL em um Estudo Integrado de Programação}

% A subtitle is optional and this may be deleted
%\subtitle{Optional Subtitle}

\author{Lucas Alves da E. Oliveira\inst{1}}
% - Give the names in the same order as the appear in the paper.
% - Use the \inst{?} command only if the authors have different
%   affiliation.

\institute[Universidade Estadual de Feira de Santana] % (optional, but mostly needed)
{
  \inst{1}%
  Departmento de Ciências Exatas\\
  Universidade Estadual de Feira de Santana
  }
% - Use the \inst command only if there are several affiliations.
% - Keep it simple, no one is interested in your street address.

\date{Seminário, 2016}
% - Either use conference name or its abbreviation.
% - Not really informative to the audience, more for people (including
%   yourself) who are reading the slides online

\subject{Produção de textos técnicos e acadêmicos}
% This is only inserted into the PDF information catalog. Can be left
% out. 

% If you have a file called "university-logo-filename.xxx", where xxx
% is a graphic format that can be processed by latex or pdflatex,
% resp., then you can add a logo as follows:

% \pgfdeclareimage[height=0.5cm]{university-logo}{university-logo-filename}
% \logo{\pgfuseimage{university-logo}}

% Delete this, if you do not want the table of contents to pop up at
% the beginning of each subsection:

%\AtBeginSubsection[]
%{
%  \begin{frame}<beamer>{Outline}
%    \tableofcontents[currentsection,currentsubsection]
%  \end{frame}
%}

% Let's get started
\begin{document}

\begin{frame}
  \titlepage
\end{frame}

%\begin{frame}{Outline}
%  \tableofcontents
%  % You might wish to add the option [pausesections]
%\end{frame}

% Section and subsections will appear in the presentation overview
% and table of contents.
\section{slide}

\subsection{Introdução}

\begin{frame}{Introdução ao Problm Based Learning}{História e sua relação com nosso contexto}
  \begin{itemize}
    \item História PBL
    \begin{itemize}
        \item \alert{Problem Based Learning}: Início na escola de medicina na Universidade McMaster (Canadá).
        \item Estratégia educacional com foco na aprendizagem de habilidades profissionais
    \end{itemize} 
    \item Adoção no meio acadêmico brasileiro
    \begin{itemize}
        \item Primeira adoção na Faculdade de Marília e Universidade Estadual de Londrina na área da medicina.
        \item Desde 2003.1 utilizado na UEFS em Engenharia de Computação e Medicina
    \end{itemize}
  \end{itemize}
\end{frame}

\subsection{Introdução 2}

% You can reveal the parts of a slide one at a time
% with the \pause command:
\begin{frame}{Metodologia PBL}
  \begin{enumerate}
  \item {
    Início: apresentação, leitura e interpretação do problema;
    \pause % The slide will pause after showing the first item
  }
  \item {   
    \textit{Brainstorming}: incentivo a criatividade sem pré-julgamento;
    \pause
  }
  \item {
    Sistematização: Organização das ideias obtidas.
    \pause
  }
  \item{
    Elaboração de questões: Encontrar os principais desafios.
    \pause
  }
  \item {
    Metas de aprendizagem: passos necessário para resolução dos desafios.
    \pause
  }
  \item{
    Avaliação: análise de aspectos que causam dificuldades.
    \pause
  }
  \item{
    Repetição: na próxima sessão, o ciclo PBL é retomado do item 2. 
  }
  \end{enumerate}
\end{frame}

\subsection{Estudo Integrado}

\begin{frame}{Módulos Integradores}
\begin{block}{Estudo Integrado}
Junção de módulo teórico e módulo com metodologia PBL sobre um certo tema
\end{block}
\end{frame}

\subsection{Estudo Integrado de Programação}
\begin{frame}{Estudo Integrado de Programação}{currículo (antigo) de Eng. de Computação - UEFS}
    Módulos Integradores de Programação:
    \begin{description}
    \item[Algoritmo e Programação II] Conceitos de POO e estruturas de dados.
    \item[Estrutura de Dados] Funções de \textit{hash}, grafos e árvores.
    \item[Estrutura Discretas] Teoria de conjunto, combinatória.
    \end{description}
\end{frame} 

\subsection{Experiência}

 \begin{frame}{Pesquisa}
\begin{block}{Metodologia de Pesquisa}
Relato de experiência de 40 alunos, divididos em 4 grupos para as sessões tutoriais
expostos também a aulas expositivas, palestras e apresentações. No período de outubro
de 2007 a abril de 2008. 
\end{block}
\end{frame}

\subsection{Problema PBL}

\begin{frame}

Uma das chaves para o sucesso da metodologia PBL são os problemas
apresentados, assim Santos destaca 5 característica relevantes
dos problemas PBL:

\begin{itemize}
\item O problema deve motivar os alunos.
\item O problema deve levar a tomada de decisões.
\item O problema deve estimular a cooperação.
\item Os problemas devem promover discussão.
\item Estimulação de metas de aprendizagem.
\end{itemize}
\end{frame}




\subsection{asasa}

\begin{frame}{Blocks}
\begin{block}{Block Title}
You can also highlight sections of your presentation in a block, with it's own title
\end{block}
\begin{theorem}
There are separate environments for theorems, examples, definitions and proofs.
\end{theorem}
\begin{example}
Here is an example of an example block.
\end{example}
\end{frame}


% Placing a * after \section means it will not show in the
% outline or table of contents.
\section{Summary}

\begin{frame}{Summary}
  \begin{itemize}
  \item
    The \alert{first main message} of your talk in one or two lines.
  \item
    The \alert{second main message} of your talk in one or two lines.
  \item
    Perhaps a \alert{third message}, but not more than that.
  \end{itemize}
  
  \begin{itemize}
  \item
    Outlook
    \begin{itemize}
    \item
      Something you haven't solved.
    \item
      Something else you haven't solved.
    \end{itemize}
  \end{itemize}
\end{frame}



% All of the following is optional and typically not needed. 
\appendix
\section<presentation>*{\appendixname}
\subsection<presentation>*{For Further Reading}

\begin{frame}[allowframebreaks]
  \frametitle<presentation>{For Further Reading}
    
  \begin{thebibliography}{10}
    
  \beamertemplatebookbibitems
  % Start with overview books.

  \bibitem{Author1990}
    A.~Author.
    \newblock {\em Handbook of Everything}.
    \newblock Some Press, 1990.
 
    
  \beamertemplatearticlebibitems
  % Followed by interesting articles. Keep the list short. 

  \bibitem{Someone2000}
    S.~Someone.
    \newblock On this and that.
    \newblock {\em Journal of This and That}, 2(1):50--100,
    2000.
  \end{thebibliography}
\end{frame}

\end{document}


