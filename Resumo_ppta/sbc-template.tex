\documentclass[12pt]{article}

\usepackage{sbc-template}

\usepackage{graphicx,url}

\usepackage[brazil]{babel}   
%\usepackage[latin1]{inputenc}  
\usepackage[utf8]{inputenc}  
% UTF-8 encoding is recommended by ShareLaTex

     
\sloppy

\title{Abordagem da Disciplina Produção de Textos Técnicos e Acadêmicos\\ (TEC503)}

\author{Lucas A. da Encarnação Oliveira\inst{1} }


\address{Departamento de Ciências Exatas -- Universidade Estadual de Feira de Santana
  (UEFS)\\
  Feira de Santana -- BA -- Brazil
  \email{lucas2213690@gmail.com}
}

\begin{document} 

\maketitle

     
\begin{resumo} 
  Este artigo irá explanar a abordagem curricular da disciplina de Produção de Textos Técnicos e Acadêmicos, no contexto do curso de Engenhariaa de Computação na Universidade Estadual de Feira de Santana e sua importância para o decorrer do curso.
\end{resumo}

% objetivos 

\section{Objetivos da TEC503}

O objetivo é habilitar os discentes a produzir textos técnicos e acadêmicos, diferenciar cada tipo de texto e suas especificidades, aprofundar-se no âmbito teórico de relatório técnico da metodologia PBL e criar um ambiente de discussão para estimular os alunos.
Para cumprir estes objetivos a disciplina recorre a conceitos de texto técnicos e acadêmicos, coesão e coerência e sua importância para um texto bem-escrito e organizado, também é utilizado a correção de relatórios com metodologia PBL com a intençã de sanar imperfeições e melhorar os relatórios posteriores e por fim uma introdução a pesquisa acadêmica e ao \LaTeX.


\section{Coesão e Coerência Textual} \label{sec:firstpage}
 
\subsection{Definição}

Um texto em si envolve normas, relação sintático-semântica e o contexto onde este está inserido, sendo o conceito de textualidade um produto dessas três caracteristícas. A disciplina TEC 503 foca na textualidade e coesão e coerência, pois possibilita
o discente escrever textos bens elaborados \cite{koch:02}.

%Coesão e Coerência

\subsection{Coerência Textual}

A coerência textual pode ser definida pela presença de características como: unidade textual (continuidade), evolução das informações apresentadas, Não-Contradição.

\begin{itemize}

	\item Unidade Textual: É a ligação de todos elementos contidas no texto, já que é sempre necessário retomar uma ideia anterior para explicar uma nova informação,
	 apresentando assim uma continuidade.
	\item Evolução: Intimamente ligada ao conceito anterior, a evolução se trata da progressão das informação ao decorrer do texto, soma das ideias apresentadas.
	\item Não-Contradiçao: O texto deve respeitar a lógica para que os dois intens anteriores sejam validos, pois a contradição invalida todo o conjunto do texto,
	para evitar isso é de suma importância conhecer o significado das palavras empregadas no texto.
	
\end{itemize}


\section{Coesão Textual}

A coesão textual é o bom emprego de ligações semânticas ao longo do texto,
 formando um texto coeso (totalidade semântica), para isso recorre-se a várias
tecnicas, uma delas é o uso de conjunções, a mais utilizada é o uso de conjunções.

%

\section{Fontes de Pesquisa}

	A fonte de Pesquisa é muito importante para refinar a busca e encontrar boas referências
	para sua produção textual, e deve-se utilizar certos critérios para filtra-las.

\subsection{Critérios}
	A Instituição do autor e nível de inovação são normalmente os maiores critérios de avaliação, e as fonte confiáveis são:
	livros, artigos, monografias, dissertações e teses. Outros críterios devem ser levados em conta como clareza, coerência e
	a atualização da obra. 
	
	Devemos evitar websites, já que geralmente a informação disponibilizada nessas ferramentas são bastante condensadas,
	sendo assim não podem embasar um trabalho acadêmico.

\subsection{Fonte de Informação}

		Um boa fonte é essencial para agilizar a busca de boas referências, e para isso podemos utilizar técnicas de refinamento de ferramentas
		de buscas, repositórios acadêmicos, revistas científicas etc.
		 
			
	


\section{Citação e Referência}

Citações são trechos parafraseados ou retirados de fontes consultadas previamente para
reforçar a argumentação do autor e embasar uma produção textual. Não confudir este conceito com plágio, já que é totalmente diferente \cite{abnt:02}.

\subsection{Tipos de citações}
Há diversas formas de utilizar as citações, elas serão
caracterizadas abaixo.

\begin{itemize}
	\item Citação: Menção ou transcrição de informações de outras fontes utilizadas para complementar a ideia do autor
	\item Citação de Citaçao: Menção direta ou indireta de informações que o autor não acessou 
		  diretamente, e sim tomou conhecimento através de outro obra
	\item Citação direta: Transcrições literais de parte de um texto consultado
	\item Citação indireta: Texto baseado em um trecho de outra fonte 
\end{itemize}

\subsection{Chamadas}

Ao fazer uma citação, você deve informar o nome do autor da obra, seguida do ano de publicação (Autor-Data) ou representação númerica.
A norma NBR 6023/2002 padroniza e regulamenta estas questões.


\subsection{Relatório PBL}

Relatório PBL é uma produção de texto técnica, que deve conter os elementos de coesão e coerência, possuir uma linguagem precisa e objetividade.

\subsection{Estrutura do Relatório}

\begin{description}
   \item[Introdução] Contextualiza e traz a motivação do projeto.
   \item[Desenvolvimento] Passos utilizados para alcançar o objetivo (Metodologia, Sessões).
   \item[Conclusão] Traz uma reflexão sobre os objetivos de aprendizagem alcançados.
   \item[Referência] Livros que embasaram o projeto e relatório.
 \end{description}

\subsection{Introdução}

A introdução situa a importância da solução do problema em um contexto atual, para isso pode-se recorrer
a exemplos. Em seguida, faz-se uma correlação entre o contexto geral e o problema PBL, por fim a descrição do problema é parafraseado, para em seguida no tópico abaixo explicar os passos e recursos utilizados na resolução do problema.

\subsection{Fundamentação Teorica}

Tem o objetivo de abordar os aspectos teóricos necessários para a resolução do problema apresentado, para isso recorre-se a livros que abordem o assunto. Estes aspectos são apontados através do uso de citações diretas e indiretas (de fontes confiáveis) e explicados de forma geral, com o objetivo de embasar os passos tomados escolhida pelo autor para atingir o problema.

\subsection{Metodologia}

A metodologia é geralmente dividida em vários subtópicos para organizaçã  das ideias, sendo que duas abordagens essenciais contidas
na Metodologia são as decisões e recursos utilizado no desenvolvimento do projeto e a explicação do projeto (funcionamento) em si.

Na primeira abordagem, é contido as discussões chaves das sessões PBL, justificativas de tomadas de decisão no projeto, a interpretação pessoal do autor para abordar o problwma, fundamentação téorica e recursos (técnias de programação) utilizados no projeto. Preferencialmente
, também dividido em subtópicos definidos e mantendo a linguagem impessoal ao longo do texto.

A segunda abordagem discute o resultado (projeto), contendo assim elementos que caracterizem por si só o projeto, como: pseudocódigos, diagramas, fórmulas matemáticas. Estes elementos devem ser detalhadamente explanados pelo autor, com uma linguagem precisa e técnica.   

\section{Pesquisa acadêmica em Engenharia de Computação}

\subsection{Tema}

Primeiramente o estudante deve delimitar um tema específico que será objeto de pesquisa, consequentemente ele irá atacar um problema da dessa área, propondo
assim uma solução através da pesquisa. Recomenda-se que o pesquisador deve possuir certa experiência e afinidade em relação ao tema e que não seja abordado um tema amplo, já que a pesquisa deve aprofundar-se em aspectos específicos.

\subsection{Projeto de Pesquisa}

É necessário a elaboração de texto acadêmico contendo uma explicação do problema teórico, a abordagem metodológica, as técnicas de pesquisa e hipótese dos possíveis resultado. Este trabalho será a base da pesquisa, contendo todos os elementos necessários para tal prática. 

\section{Introdução ao \LaTeX}

O \LaTeX é um sistema tipográfico, amplamente utilizado em ambientes científicos, criado em 1982 por Donald E. Knuth \cite{oetiker:98}.

\subsection{Vantagens}

\begin{itemize}
	\item Alta qualidade tipográfica
	\item Foco no conteúdo do texto
	\item Estrutura Rígida (Evita Erros)
\end{itemize}

\subsection{Desvantagens}

\begin{itemize}
	\item Curva de Aprendizagem
	\item Mensagens de erro não-claras
\end{itemize}


\bibliographystyle{sbc}
\bibliography{sbc-template}

\end{document}
